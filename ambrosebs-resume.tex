% LaTeX source of my resume
% =========================

% Commented for easy reuse... ;)

% See the `README.md` file for more info.

% This file is licensed under the CC-NC-ND Creative Commons license.


% Start a document with the here given default font size and paper size.
\documentclass[10pt,a4paper]{article}

% Set the page margins.
\usepackage[a4paper,margin=0.75in]{geometry}

% Setup the language.
\usepackage[english]{babel}
\hyphenation{Some-long-word}

% Makes resume-specific commands available.
\usepackage{resume}




\begin{document}  % begin the content of the document
\sloppy  % this to relax whitespacing in favour of straight margins


% title on top of the document
\maintitle{Ambrose Bonnaire-Sergeant}{Last update on \today}

\nobreakvspace{0.3em}  % add some page break averse vertical spacing

% \noindent prevents paragraph's first lines from indenting
% \mbox is used to obfuscate the email address
% \sbull is a spaced bullet
% \href well..
% \\ breaks the line into a new paragraph
\noindent\href{mailto:abonnairesergeant.at.gmail.dot.com}{abonnairesergeant\mbox{}[at]\mbox{}gmail.com}\sbull
%\textsmaller{+}31.646469087\sbull
\href{http://ambrosebs.com/}{ambrosebs.com}\sbull
\href{https://github.com/frenchy64}{github.com/frenchy64}
\\
Lindley Hall, Room 315\sbull
150 S Woodlawn Ave \sbull
Bloomington \sbull 
IN 47405 \sbull
USA


\spacedhrule{0.9em}{-0.4em}  % a horizontal line with some vertical spacing before and after

\roottitle{Summary}  % a root section title

\vspace{-1.3em}  % some vertical spacing
\begin{multicols}{2}  % open a multicolumn environment
\noindent 
\emph{Programming language tooling developer, bringing academia to the masses.}
\\
\\
I love delivering tools and lessons from academia to working programmers.
My work in building optional type systems has been incredibly
rewarding, and fulfills a need of mine to make a real impact
on other people's lives.

My work on Typed Clojure is indicative of my approach.
I created the optional type system for Clojure
in 2011.
Each subsequent year has taken a new direction
based on the feedback of the community, using my 
expertise to match solutions to problems.
\\

I love the communicating and marketing ideas.
It is both rewarding and challenging.
I have orchestrated crowdfunding campaigns that have
collectively raised
over \$50,000 to support my goals in open source 
development---supported by companies like Cognitect, Walmart Labs, Prismatic,
and CircleCI.
\\

I have given many talks in industry and academic conferences, including
Strangeloop, Clojure Conj, Philly ETE, Lambda Jam Brisbane, Code
Mesh, ESOP, and the International Lisp Conference.
I also enjoy talking at local Clojure user groups.
\end{multicols}


\spacedhrule{0em}{-0.4em}

\roottitle{Experience}

\headedsection
  {{Research Assistant}}
  {\textsc{Indiana University Bloomington}} {%
  \headedsubsection
    {Advisor: Sam Tobin-Hochstadt}
    {Fall 2014 --- Summer 2015, Spring 2016}
    {\bodytext{
        We have worked to formalise and publish a paper on the fundamentals
        of Typed Clojure.
        We are also working towards extending Typed Clojure with gradual typing.
        More recently, we are developing a
        framework to infer type annotations for optional type systems
        based on \href{http://ambrosebs.com/auto-ann.html}{runtime instrumentation}.
      }
    }
}

\headedsection
  {{Assistant Instructor}}
  {\textsc{Indiana University Bloomington}} {%
  \headedsubsection
    {Instructors: Sam Tobin-Hochstadt, Suzanne Menzel}
    {Fall 2015, Fall 2016 ---}
    {\bodytext{Taught undergraduates C211, an introductory programming
      language course based on How To Design Programs.
			Also C343, introductory data structures.
    Conducted labs and office hours.}}
}

\headedsection
  {\href{http://ambrosebs.com/\#gsoc}{Google Summer of Code}} {\textsc{Clojure organisation}} {
  \headedsubsection
    {Student} {2012, 2013}
    %{\bodytext{
    %  I developed Typed Clojure, improving the support of
    %  Clojure idioms, supporting more expressive types,
    %  and adding documentation.
    %  }
    %}
    {}
  \headedsubsection
    {Mentor} {2014 (3 students), 2015 (2 students)}
    {}
    %{\bodytext{
    %    I have mentored 5 projects, including several adding
    %    advanced type system features to Typed Clojure,
    %    and work towards a ClojureScript compiler using
    %    tools.analyzer as a backend.
    %  }}
    %{}
  \headedsubsection
    {Administrator} {2014, 2015}
    {}
    %{\bodytext{
    %  My role as an administrator included proposing
    %  projects, preparing and reviewing application documents,
    %  and advertising for interested students.
    %  }}
    %{}
}

\headedsection
  {Crowdfunded Open Source work}
  {\textsc{Perth, Western Australia}} {%
  \headedsubsection
    {\href{https://www.indiegogo.com/projects/typed-clojure}{Typed Clojure}
    %(\$35,254 USD raised by 545 backers)
  }
    {2013}
    {}
    %{\bodytext{
    %  These funds helped support me as I worked on improving 
    %  and extending Typed Clojure's type system.
    %  \$5,000 of these funds were used to commission
    %  further open source work on tools.analyzer, a
    %  now-important Clojure library.
    %}}

  \headedsubsection
  {\href{https://www.indiegogo.com/projects/gradual-typing-for-clojure}{Gradual Typing for Clojure}
    %(\$11,695 USD raised by 199 backers)
  }
    {2015}
    {}
    %{\bodytext{These funds helped me design a \emph{gradual typing} framework for Typed Clojure.
    %  %which is still in development. 
    %  It supports automatic contracts for
    %  global variables based on the typed-untyped boundary, as well as allowing
    %  arbitrary exporting of macros from typed namespaces---a shortcoming of previous systems.
    %  }}

  \headedsubsection
  {\href{https://www.indiegogo.com/projects/typed-clojure-clojure-spec-auto-annotations}{Automatic
    Annotations for Typed Clojure and clojure.spec}
    %(\$8,621 USD raised by 69 backers)
  }
    {2016}
    {}
    %{\bodytext{
    %  This campaign concentrated on automating the manual labour of type and spec annotations.
    %  It helped me attend multiple industry conferences to meet real users
    %  of Typed Clojure and clojure.spec to discuss the needs of the community.
    %  }}
}


\headedsection
  {Analyst Programmer}
  {\textsc{University of Western Australia}} {
  {2010 --- 2011}
  {\bodytext{
    I worked with the UWA library staff to gather 
    information on the format and location of various University research
    data sources, and set up VIVO semantic web software. I used Java, XSLT,
    and bash, and administered several Linux machines.
}
  }
}

\roottitle{Education}

\headedsection
  {\href{http://www.indiana.edu}{Indiana University Bloomington}}
  {\textsc{Bloomington, Indiana}} {%
  \headedsubsection
    {Computer Science PhD student (In progress)}
    {Fall 2014 ---}
    {}
}

\headedsection
  {\href{http://www.indiana.edu}{Indiana University Bloomington}}
  {\textsc{Bloomington, Indiana}} {%
  \headedsubsection
    {Master of Science in Computer Science}
    {2014 --- 2017}
    {}
}

\headedsection
  {\href{http://www.uwa.edu.au}{University of Western Australia}}
  {\textsc{Western Australia, Australia}} {%
  \headedsubsection
    {BSc in Computer Science with Honours}
    {2008 --- 2013}
    {}
}

%\roottitle{Research Groups}
%
%\headedsection
%  {\href{http://wonks.github.io/}{PL Wonks}}
%  {\textsc{Indiana University Bloomington}} 
%
%\headedsection
%  {\href{http://prl.ccs.neu.edu/gtp/index.html}{Gradual Typing Group}}
%  {\textsc{Indiana University Bloomington}} 
%
\roottitle{Publications}

\headedsection
  {\href{http://frenchy64.github.io/papers/esop16-short.pdf}{Practical Optional Types for Clojure}
  (ESOP16)}
  {\textsc{with Rowan Davies, Sam Tobin-Hochstadt}} {%
    {}
}
\headedsection
  {\href{https://s3.amazonaws.com/github/downloads/frenchy64/papers/ambrose-honours.pdf}{A Practical Optional Type System for Clojure} (Honours dissertation)}
  {\textsc{supervised by Rowan Davies}} {%
    {}
}

%\spacedhrule{1.6em}{-0.4em}

%\vspace{0.5em}
%\inlineheadsection
%  {Natural languages:}
%  {Dutch \emph{(mother tongue)}, English \emph{(full professional proficiency)}, German \emph{(limited working proficiency)}, French \emph{(elementary proficiency)} and Mandarin Chinese \emph{(beginner)}.}
%

%\spacedhrule{1.6em}{-0.4em}


\roottitle{Selected Open Source Contributions}
\headedsection
  {\href{https://github.com/clojure/core.typed}{core.typed}}
  {\textsc{Lead Developer}} {%
    {
     % \bodytext{
     %   The main library of Typed Clojure.
     % }
    }
}
\headedsection
  {\href{https://github.com/clojure/core.match}{core.match}}
  {\href{https://github.com/clojure/core.match/commits?author=frenchy64}{\textsc{Contributor}}} {%
    {%\bodytext{
     %   Studied optimising pattern matching literature and collaborated
     %   on a pattern matcher for Clojure.
     % }
    }
}

\headedsection
  {\href{https://github.com/clojure/clojure}{Clojure}}
  {\href{https://github.com/clojure/clojure/commits?author=frenchy64}{\textsc{Contributor}}} {%
    {%\bodytext{
     % Contributed several minor patches improving error messages,
     % fixing bugs, and adding features.
     % }
    }
}

\roottitle{Selected Talks}

\headedsection
  {Typed Clojure: From Optional to Gradual Typing (\href{https://www.youtube.com/watch?v=yG9CffLlXx0}{Video})}
  {\textsc{Strangeloop 2015}} {%
}
\headedsection
  {Typed Clojure in Practice (\href{https://www.youtube.com/watch?v=a0gT0syAXsY}{Video})}
  {\textsc{Strangeloop 2014}} {%
}
\headedsection
  {Practical Optional Types for Clojure (\href{http://ambrosebs.com/talks/esop16.pdf}{Slides})}
  {\textsc{ESOP 2016}} {%
  }
\headedsection
  {Introduction to Logic Programming (\href{https://www.youtube.com/watch?v=irjP8BO1B8Y}{Video})}
  {\textsc{Clojure Conj 2011}} {%
}

\roottitle{Skills}

\inlineheadsection  % special section that has an inline header with a 'hanging' paragraph
  {Technical expertise:}
  {Experienced with open source development and maintenance, 
    including Git, GitHub, and continuous integration.
    Enjoys building static and dynamic
    program verification tools for dynamically typed languages.
    Experienced Clojure programmer.
    Some experience in Java, C, Racket, and Scheme.
  }

\roottitle{Personal}

\headedsection{Interests}
{\bodytext
  {Singing, guitar, clarinet, reading, teaching.}
}


\end{document}
